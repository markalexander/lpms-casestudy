\documentclass[twocolumn]{article}
\usepackage[utf8]{inputenc}
\usepackage[english]{babel}
%\usepackage{amsmath,amssymb}
\usepackage{csvsimple}
\usepackage{tikz}
\usepackage{pgfplots}
%\usepackage{float}
\pgfplotsset{compat=1.14}
%\usepackage{xcolor}
\usepackage{graphicx}
\usepackage[textsize=tiny]{todonotes}


\usepackage{pgfplotstable}

\newcommand{\chartheight}{7cm}
\newcommand{\chartlegendpos}{(0.5,-0.2)}

\definecolor{cGas}{RGB}{202,178,214}
\definecolor{cCoal}{RGB}{180,180,180}
\definecolor{cNuclear}{RGB}{253,191,111}
\definecolor{cWind}{RGB}{178,223,138}
\definecolor{cHydro}{RGB}{166,206,227}
\definecolor{cInterconnect}{RGB}{251,154,153}


%
% GENERATION SCHEDULE
%

\newcommand{\chartgenschedstacked}[1]{%
    \pgfplotstableread[col sep=comma]{#1}\loadedtable
    \begin{tikzpicture}
        \begin{axis}[
            ymin=0,
            minor tick num=4,
            enlarge x limits=false,
            const plot,
            axis on top,
            stack plots=y,
            cycle multi list={%
                {cWind!60!black,fill=cWind},%
                {cNuclear!60!black,fill=cNuclear},%
                {cGas!60!black,fill=cGas},%
                {cCoal!60!black,fill=cCoal},%
                {cInterconnect!60!black,fill=cInterconnect},%
                {cHydro!60!black,fill=cHydro},%
             },
            xlabel={Time (h)},
            ylabel={Generation (GW)},
            width=\columnwidth,
            height=\chartheight,
            legend style={
                area legend,
                at={\chartlegendpos},
                anchor=north,
                legend columns=3}
            ]

        \addplot table[x=StartTime,y expr=\thisrow{Wind} * 0.001] from \loadedtable \closedcycle;
        \addplot table[x=StartTime,y expr=\thisrow{Nuclear} * 0.001] from \loadedtable \closedcycle;
        \addplot table[x=StartTime,y expr=\thisrow{Gas} * 0.001] from \loadedtable \closedcycle;
        \addplot table[x=StartTime,y expr=\thisrow{Coal} * 0.001] from \loadedtable \closedcycle;
        \addplot table[x=StartTime,y expr=\thisrow{Interconnect} * 0.001] from \loadedtable \closedcycle;
        \addplot table[x=StartTime,y expr=\thisrow{Hydro} * 0.001] from \loadedtable \closedcycle;
        \legend{Wind, Nuclear, Gas, Coal, Intercon., Hydro}
        \end{axis}
    \end{tikzpicture}%
}


%
% CO2 RAMPDOWN
%

\newcommand{\chartrampdown}{%
\pgfplotstableread[col sep=comma]{output/co2rampdown.csv}\loadedtable
\begin{tikzpicture}

	\begin{axis}[
        width=0.9\columnwidth,
        height=\chartheight,
		ybar stacked,
		axis y line*=left,
		ylabel={Daily Output (GWh)},
		xlabel={Reduction in CO2 Emissions Limit (\%)},
        ymin=0,
        legend style={
          area legend,
          at={\chartlegendpos},
          anchor=north,
          legend columns=3}
	]
	\addplot[cWind!60!black,fill=cWind] table[x=Reduction,y expr=\thisrow{Wind}/1000]{\loadedtable};
	\addplot[cNuclear!60!black,fill=cNuclear] table[x=Reduction,y expr=\thisrow{Nuclear}/1000]{\loadedtable};
	\addplot[cGas!60!black,fill=cGas] table[x=Reduction,y expr=\thisrow{Gas}/1000]{\loadedtable};
	\addplot[cHydro!60!black,fill=cHydro] table[x=Reduction,y expr=\thisrow{Hydro}/1000]{\loadedtable};
	\addplot[cCoal!60!black,fill=cCoal] table[x=Reduction,y expr=\thisrow{Coal}/1000]{\loadedtable};
	\addplot[cInterconnect!60!black,fill=cInterconnect] table[x=Reduction,y expr=\thisrow{Interconnect}/1000]{\loadedtable};
        \legend{Wind, Nuclear, Gas, Hydro, Coal, Intercon.}
   \end{axis}

	\begin{axis}[
        width=0.9\columnwidth,
        height=\chartheight,
		axis y line*=right,  
		axis x line = none,
		ylabel={Profit (\textsterling m)}]
			\addplot table[x=Reduction,y expr=\thisrow{Profit}] from \loadedtable;
	\end{axis}  

	\end{tikzpicture}
}


%
% CO2 CREDIT
%

\newcommand{\chartcredit}[1]{%
\pgfplotstableread[col sep=comma]{#1}\loadedtable
\begin{tikzpicture}

	\begin{axis}[
        width=0.9\columnwidth,
        height=\chartheight,
		ybar stacked,
		axis y line*=left,
		xlabel={Carbon Price (\textsterling)},
		ylabel={Daily Output (GWh)},
        bar width=4pt,
        xmin=-5,
        xmax=105,
        ymin=0,
        legend style={
          area legend,
          at={(0.5,-0.2)},
          anchor=north,
          legend columns=3}
	]
	\addplot[cWind!60!black,fill=cWind] table[x=Co2Price,y expr=\thisrow{Wind}/1000]{\loadedtable};
	\addplot[cNuclear!60!black,fill=cNuclear] table[x=Co2Price,y expr=\thisrow{Nuclear}/1000]{\loadedtable};
	\addplot[cGas!60!black,fill=cGas] table[x=Co2Price,y expr=\thisrow{Gas}/1000]{\loadedtable};
	\addplot[cCoal!60!black,fill=cCoal] table[x=Co2Price,y expr=\thisrow{Coal}/1000]{\loadedtable};
	\addplot[cHydro!60!black,fill=cHydro] table[x=Co2Price,y expr=\thisrow{Hydro}/1000]{\loadedtable};
	\addplot[cInterconnect!60!black,fill=cInterconnect] table[x=Co2Price,y expr=\thisrow{Interconnect}/1000]{\loadedtable};
        \legend{Wind, Nuclear, Gas, Coal, Hydro, Intercon.}
   \end{axis}

	\begin{axis}[
        width=0.9\columnwidth,
        height=\chartheight,
		axis y line*=right,  
		axis x line = none,
        xmin=-5,
        xmax=105,
		ylabel={Profit (\textsterling m)}
		]
			\addplot table[x=Co2Price,y expr=\thisrow{Profit}/1000000] from \loadedtable;
	\end{axis}  

	\end{tikzpicture}
}


%
% CARBON TAX
%

\newcommand{\chartcarbontax}[1]{%
\pgfplotstableread[col sep=comma]{#1}\loadedtable
\begin{tikzpicture}

	\begin{axis}[
        width=0.9\columnwidth,
        height=\chartheight,
		ybar stacked,
		axis y line*=left,
		xlabel={Carbon Tax (\textsterling)},
		ylabel={Daily Output (GWh)},
        bar width=4pt,
        xmin=-5,
        xmax=105,
        ymin=0,
        legend style={
          area legend,
          at={(0.5,-0.2)},
          anchor=north,
          legend columns=3}
	]
	\addplot[cWind!60!black,fill=cWind] table[x=Co2Price,y expr=\thisrow{Wind}/1000]{\loadedtable};
	\addplot[cNuclear!60!black,fill=cNuclear] table[x=Co2Price,y expr=\thisrow{Nuclear}/1000]{\loadedtable};
	\addplot[cGas!60!black,fill=cGas] table[x=Co2Price,y expr=\thisrow{Gas}/1000]{\loadedtable};
	\addplot[cCoal!60!black,fill=cCoal] table[x=Co2Price,y expr=\thisrow{Coal}/1000]{\loadedtable};
	\addplot[cHydro!60!black,fill=cHydro] table[x=Co2Price,y expr=\thisrow{Hydro}/1000]{\loadedtable};
	\addplot[cInterconnect!60!black,fill=cInterconnect] table[x=Co2Price,y expr=\thisrow{Interconnect}/1000]{\loadedtable};
        \legend{Wind, Nuclear, Gas, Coal, Hydro, Intercon.}
   \end{axis}

	\begin{axis}[
        width=0.9\columnwidth,
        height=\chartheight,
		axis y line*=right,  
		axis x line = none,
        xmin=-5,
        xmax=105,
		ylabel={Profit (\textsterling m)}
		]
			\addplot table[x=Co2Price,y expr=\thisrow{Profit}/1000000] from \loadedtable;
	\end{axis}  

	\end{tikzpicture}
}




%
% GEN SCHEDULE TABLE
%


\newcommand{\productiontable}[1]{%
    \begin{tabular}{lcccccc}
		\hline
		\textbf{Period} & \textbf{Gas} & \textbf{Coal} & \textbf{Nuclear} & \textbf{Wind} & \textbf{Hydro} & \textbf{Interconnect} \\ \hline
		\csvreader[head to column names]{#1}{}
		{\Period & \Gas & \Coal & \Nuclear & \Wind & \Hydro & \Interconnect \\} \\ \hline
    \end{tabular}
}




%%%%%%%%%%%%%%%%%%%%%%%%%%%%%%%%%%%%%%%%%%%%%%%%%%%%%%%%%%%%%%%%%%%%%%%%
%%%%%%%%%%%%%%%%%%%%%%%%%%%%%%%%%%%%%%%%%%%%%%%%%%%%%%%%%%%%%%%%%%%%%%%%
%%%%%%%%%%%%%%%%%%%%%%%%%%%%%%%%%%%%%%%%%%%%%%%%%%%%%%%%%%%%%%%%%%%%%%%%






\newcommand{\chartgencarbonprice}[2]{%
    \pgfplotstableread[col sep=comma]{#2}\loadedtable
    \begin{tikzpicture}
        \begin{axis}[
        	title={#1},
            ymin=0,
            minor tick num=4,
            enlarge x limits=false,
            const plot,
            axis on top,
            stack plots=y,
            cycle multi list={%
                {cGas!60!black,fill=cGas},%
                {cCoal!60!black,fill=cCoal},%
                {cNuclear!60!black,fill=cNuclear},%
                {cWind!60!black,fill=cWind},%
                {cHydro!60!black,fill=cHydro},%
                {cInterconnect!60!black,fill=cInterconnect},%
             },
            xlabel={Time (h)},
            ylabel={Generation (MW)},
            width=\columnwidth,
            height=7cm,
            legend style={
                area legend,
                at={(0.5,-0.2)},
                anchor=north,
                legend columns=3}
            ]

        \addplot table[x=StartTime,y=Gas] from \loadedtable \closedcycle;
        \addplot table[x=StartTime,y=Nuclear] from \loadedtable \closedcycle;
        \addplot table[x=StartTime,y=Coal] from \loadedtable \closedcycle;
        \addplot table[x=StartTime,y=Wind] from \loadedtable \closedcycle;
        \addplot table[x=StartTime,y=Hydro] from \loadedtable \closedcycle;
        \addplot table[x=StartTime,y=Interconnect] from \loadedtable \closedcycle;
        %\addplot+[stack plots=false] table[x=time,y=memtotal]     from \loadedtable;
        \legend{Gas, Coal, Nuclear, Wind, Hydro, Interconnect}
        \end{axis}
	\end{axis}  
    
    \end{tikzpicture}%
}



%############################################################
% Here is Matthew's very badly done graph of Hydro, Pump, and Reserve
% #####

% Color theme
\definecolor{cHydroReserve}{RGB}{253,191,111}
\definecolor{cPump}{RGB}{178,223,138}
\definecolor{cHydro}{RGB}{166,206,227}

\pgfplotsset{
    ylabel right/.style={
        after end axis/.append code={
            \node [rotate=90, anchor=north] at (rel axis cs:1,0.5) {#1};
        }   
    }
}


\newcommand{\charthydro}[1]{%
    \pgfplotstableread[col sep=comma]{#1}\loadedtable
    \begin{tikzpicture}
        \begin{axis}[
            ymin=0,
            minor tick num=4,
            enlarge x limits=false,
            const plot,
            axis on top,
            xlabel={Time (h)},
            ylabel={Power (GW)},
            width=\columnwidth,
            height=7cm,
            ylabel right=Energy (GWh),
            legend style={
                at={(0.5,-0.2)},
                anchor=north,
                legend columns=3}
            ]

        \addplot table[x=StartTime,y expr = \thisrow{Hydro} * 0.001] from \loadedtable;
        \addplot table[x=StartTime,y expr = \thisrow{Pump} * 0.001] from \loadedtable;
        %\addplot table[x=StartTime,y=HydroReserve] from \loadedtable; % old reservoir level plot
        % new reservoir level plot
        \addplot[
    sharp plot,
    mark = square*,
    ]
    coordinates {
    (0,0.800)(6,12.800)(8,13.600)(16,9.600)(20,3.200)(22,0)(24,0.800)
    };
        %\addplot+[stack plots=false] table[x=time,y=memtotal]     from \loadedtable;
        \legend{Discharge, Pump, Reserve Level}
        \end{axis}
            
    \end{tikzpicture}%
}
% #######################
% End of badly done graph
% #######################



% Hydro table Matthew
\newcommand{\hydrotable}[1]{%
    \begin{tabular}{lccc}
		\hline
		\textbf{Period} & \textbf{Hydro} & \textbf{Pump} &\textbf{Reserve}
        \\ \hline
		\csvreader[head to column names]{#1}{}
		{\Period & \Hydro & \Pump & \HydroReserve \\} \\ \hline
    \end{tabular}
}


\author{Mark Alexander, Abby Li,\texorpdfstring{\\}{}Szymon Sacher, and Matthew Yau}
\date{April 2017}

\usepackage{enumitem}
\usepackage{pdflscape}
\usepackage{float}

\geometry{margin=0.9in}

\title{\Huge EPower\ Case Study}
\author{Mark Alexander, Abby Li,\\[0.2cm]Szymon Sacher, Matthew Yau}


\begin{document}

	\begin{titlepage}
	\clearpage
	\maketitle
    \thispagestyle{empty}

    \end{titlepage}
    
    % 
    %  OVERVIEW
    % 
    
    \section{Overview}
    
    EPower is a national electricity provider for a small country.  It relies on various energy sources--gas, coal, nuclear, wind, hydro and an external interconnect--in order to meet the daily demands of its customers.
    
    This study has been commissioned with the aim of assisting EPower in planning for a potential reduction in carbon dioxide emission limits, as well as developing general strategies for coping with variable consumer demand and wind power availability.  The overall objective throughout is that of maximizing EPower's total profit whilst satisfying power demand.
    
    
    % 
    % CURRENT OPERATIONS
    % 
    
    \section{Current Operations}
	
    In order to investigate the proposed scenarios, we must first examine the base case.  To this end, we build a model which replicates EPower's current day-to-day operations. The primary considerations here include: the level of demand for electricity from consumers; the generation capacities, costs, and limitations of each electricity source; regulatory limits on emissions that arise from the generation activities; and the price charged for electricity sold.  Demand for electricity varies throughout the day, which is split into six variable-length periods.  The objective is to maximise profit whilst meeting consumer demand and satisfying the other constraints.  This is done by making appropriate decisions about how much to generate and vary output for each energy source, as well as how best to utilise EPower's hydro reserve facilities. Full details of the basic model parameters can be found in Appendix \ref{sec:params} and in the attached \texttt{Xpress-Mosel} model file.
    
    The optimal generation schedule under this base model is shown by Figure \ref{fig:base-optgensched}, and more precisely by Table \ref{table:basic-optgensched} in Appendix \ref{sec:app-currops}.  In adhering to this policy, EPower could expect an average daily profit of approximately \textsterling 2 million, consisting of \textsterling 15 million in revenue less \textsterling 13 million in running costs and \textsterling 0.4 million in increase costs.
    
    \begin{figure}[ht!]
        \centering
        \chartgenschedstacked{output/base.csv}
        \caption{Optimal generation schedule for the base model\label{fig:base-optgensched}}
    \end{figure}
    
    Examining the schedule in more detail, we note that most power sources do not operate at maximum capacity.  The key exceptions to this are wind and hydro.  The former, being by far the least expensive source of energy, is run at maximum capacity throughout the day.  This is a theme that will be seen throughout our analysis, since it is almost always advantageous to take advantage of such an cheap and flexible resource.  The hydro system, on the other hand, is used in bursts to meet demand spikes during peak hours.  Water is pumped into the reservoir during quieter periods to be discharged later in the day, reaching full output capacity during the hours of 4pm-8pm. The operation of the hydro system in the base model is further described by Table \ref{table:base-hydrosched} and Figure \ref{fig:base-hydrosched} in Appendix \ref{sec:app-currops}.
    
	Another key observation is that nuclear remains constant throughout the generation schedule.  This is due to the fact that changing the output level here comes at great cost.  To a (much) lesser extent, this is also true of gas, which tends to stay relatively constant unless a restriction in capacity elsewhere requires it.
        
	Finally, we note that the interconnect is similar to hydro in that it can be used to provide additional capacity in bursts to satisfy peak demand, with the caveat that it is far more expensive.  For that reason it should be used sparingly and does not feature in the base model's optimal generation schedule.
    
    Before moving on to the detailed scenarios, we provide the following general advice:
    \begin{itemize}
    
    	\item The marginal (fair) value of extra wind capacity is significantly positive in every period.  In this sense, the base model shows clear signs that an investment in wind capacity would likely be beneficial.
        
        \item Perhaps counter-intuitively, demand is of \textit{negative} value in every period.  As things are, selling more electricity would actually result in a reduction in profit.  It may even be worth reducing some of the existing demand by e.g. re-negotiating contracts or reducing sales efforts and letting natural attrition take place.
    
    \end{itemize}
    
    
    
    % 
    % VARIABLE DEMAND AND WIND AVAILABILITY
    % 
    
    \section{Variability of Demand and Wind Availability}
    
    The level of consumer demand and of available wind power can vary considerably from day-to-day.  Any significant change in either factor will necessitate a change in the optimal policy.  Here, we have been asked to examine the effects of a 20\% increase in both demand and maximum wind output capacity on EPower's operations.
    
    Under this new model, a significant loss of approximately \textsterling 1 million is incurred.  The optimal generation schedule is shown by Figure \ref{fig:variable-optgensched} and Table \ref{table:variable-optgensched} (Appendix \ref{sec:app-variable}).
    
    \begin{figure}[ht!]
        \centering
        \chartgenschedstacked{output/multipliers.csv}
        \caption{Optimal generation schedule when both demand and wind are increased by 20\% \label{fig:variable-optgensched}}
    \end{figure}
    
   The primary cause of the loss is simply a greater reliance on more expensive sources of energy.  Most notably, peak hours require the use of the interconnect in order to meet demand.  As the most expensive source by running cost, this quickly eats into profits and dwarfs the effects of the relatively modest increases seen in the other sources.
    
    \subsection{Mitigation Strategies}
    	
        For this specific scenario, we make two main recommendations.
        
        The first is to attempt to flatten out demand throughout the day in general.  Reducing variability would lessen the strain on resources and hence curb the reliance on the most expensive forms of energy.  In order to do this we suggest incentivising off-peak usage with a two-pronged approach by type of consumer:
        
        \begin{itemize}
        	
            \item For large industrial clients (e.g. chemical processing plants, refineries, and other manufacturers): build relationships and negotiate agreements that offer reduced rates for running energy-intensive or batch processes during off-peak hours.
 			
            \item For domestic customers: introduce \textit{differential tariffs} to discourage peak and encourage off-peak usage.  Various well-tested economic models (e.g. price elasticity of demand) could be employed to determine an optimal pricing structure.  An example of this approach can be seen in the \textit{Economy 7} scheme run by some energy suppliers in the United Kingdom.
            
        \end{itemize}
        
        The second of our recommendations is to investigate the viability of expanding hydro output capacity.  This should move peak capacity away from the expensive interconnect and provide considerable savings.  In this case, during the fourth period (4pm-8pm), hydro operates at full capacity and the marginal value of additional output is \textsterling 45. If additional capacity can be procured for less than this amount, it would likely be beneficial to do so, at least under the assumption that other multiplier increases result in a similar generation schedule.
    
    More generally, optimal policy will vary by the exact amount of the increase in demand and wind availability.  It is hard, therefore, to determine a general optimal generation policy given only one such scenario.  A more comprehensive approach would involve simulating the effect on profit using the \textit{distribution} of demand and wind inputs in order to obtain an optimal approach over all reasonable scenarios.  Ideally, operating policy would be adjusted dynamically in response changing conditions throughout the day, though this may not be entirely feasible for certain sources and circumstances.  Deeper analysis and consultancy on these aspects is available if desired.
    
    
    % 
    %  REDUCED CO2
    % 
    
    \section{Reduced Carbon Dioxide Emissions Limits}
    
    It is expected that carbon dioxide emissions limits are to be reduced by up to 50\% in the coming years.  Here, we have been tasked with examining the effect that these reductions would have on EPower's operations and general profitability. We aim to provide a number of strategies to mitigate the effects of these new limits, as well as to provide guidance within a wider context.
    
    As a primer, we first look at how a reduction in emissions limits would affect operations given the current infrastructure and capacity setup.  These results are summarised in Figure \ref{fig:co2rampdown} and detailed in Table \ref{table:co2-rampdown} (Appendix \ref{sec:app-co2}). As can be seen, the outcome is largely unfavourable. Once carbon limits have been reduced by 25-30\%, EPower's operations will begin yielding a net loss.  At the full 50\% reduction, a loss of approximately \textsterling 2.5 million is incurred.  As this figure represents the loss incurred in an \textit{average} day of operations, it is clear that the long-term losses involved would be severe.
    
     \begin{figure}[ht!] 
        \centering
        \chartrampdown
    	\caption{Optimal daily source output and profit for decreasing CO\textsubscript{2} emission limit\label{fig:co2rampdown}}
    \end{figure}
    
    The causes of this sharp deterioration in profits can be seen in the change in the composition of the daily output by source in Figure \ref{fig:co2rampdown}.  On the left-hand-side of the chart, we see that coal makes a considerable contribution to the daily output.  Coal is a relatively cheap source of energy, but is also one of the worst emitters of carbon.  As we move towards a larger reduction in emissions limits, coal is gradually phased out.  This leaves a gap which is filled by the other sources operating at full capacity, and worse still the introduction of a significant amount of expensive interconnect utilisation, which quickly pushes operations into a daily loss.
    
    Clearly, action must be taken to avoid this outcome.  We examine a number of aspects and approaches here.
    
    
    \subsection{Investing in Enhancements}
    
    Before moving on to our main focus of capacity investment, it might also be worth considering an investment in improving the emissions properties of EPower's current sources.  Carbon capture technology, for example, has advanced considerably in recent years, and may prove cost effective in the long-term if it allows for a return to cheaper forms of fuel despite the reduced emissions limits.  In general, this would depend on a number of factors including the size of initial investment and the nature of EPower's facilities.  Further examination of this option would be worthwhile.
    
    
    \subsection{Investing in Capacity}
    
    A straightforward strategy for dealing with the reduced emissions limits is simply to invest in additional capacity, particularly of those sources which strike a favourable balance of output, emissions, and value.
    
    In particular we have asked to consider a 1 GW capacity expansion in one of EPower's current sources.  Our analysis is detailed in Appendix \ref{sec:app-co2}.  We see that the optimal allocation of the 1 GW lies in additional wind capacity, closely followed by nuclear.  Each of these actions will result in a reduction of operational losses to approximately \textsterling 1.4 million. 
    
    Since this still represents a considerable loss, and there is in fact no 1 GW investment that will result in a return to profit, we have taken the liberty of conducting some further analysis.  The details can be found in Appendix \ref{sec:app-co2}.  From these results, we see that break-even will occur with a 2.5 GW increase in capacity for either wind or nuclear.  Investing beyond this point in nuclear does not yield any significant reward, but importantly a 4 GW investment in wind would be more than sufficient to completely recover current profit levels.

    
    \subsection{Relaxing the Limit on CO\textsubscript{2} Emissions}
    
    As mentioned above the restrictive emissions limits constrain the use of cheaper domestic power sources, i.e. gas and coal. As an alternative approach, we recommend that EPower consider strategies for relaxing this constraint in order to return to profit.  We suggest two key solutions: negotiating carbon offsetting schemes or purchasing additional permits from an appropriate emissions market.  Under the assumption that EPower will not invest in new capacity, and under the more restrictive emissions regime, the fair price of emissions permits will be approximately \textsterling 61 per unit. If the cost of offsetting or the price of CO\textsubscript{2} emissions permits was lower, the company would be able to reduce its losses through these mechanisms. Figure \ref{fig:carbon_credits} presents the optimal average daily generation schedules and the resulting profits for various prices of emissions permits, assuming that 100,000 permits will be allocated to EPower at no cost. Notably, for prices below \textsterling 24 the company will be able to make daily profit. If the price exceeds \textsterling 75, the company should only use its clean power sources (wind, nuclear and hydro), import the shortage through the interconnect and sell all of its allocated emission permits. For detailed analysis see Appendix \ref{sec:app-carbon}.
        
        \begin{figure}[!ht]
       		 \centering
        	\chartcredit{output/carbon_credit.csv}
            \caption{Profit and optimal production schedule given CO\textsubscript{2} credit price.\label{fig:carbon_credits}}
        \end{figure}
        
    
    \subsubsection{Planning for Carbon Tax}
    
    As a result of Paris Agreement, we expect that the current regime of emissions quotas will be replaced by a ``carbon tax".  Since many of the considerations are similar, it would be beneficial for EPower to formulate a carbon strategy which is also compatible with this future regime.  We have been able to determine that, given current power generation infrastructure, demand and prices, EPower will be able to make a profit only if the tax is lower than \textsterling 10 per unit of carbon emission. The optimal generation schedule and the expected profit resulting from various tax levels are presented in Figure \ref{fig:carbon_tax}, and in more detail in Appendix \ref{sec:app-tax}. Importantly, the analysis has not taken into consideration neither the likely changes in demand and prices, nor the planned investment. We are equipped to provide full analysis and advice on the most efficient investment strategy if desired.
    
    \begin{figure}[!ht]
        \centering
        \chartcarbontax{output/carbon_tax.csv}
        \caption{Profit and optimal production schedule given carbon tax level.\label{fig:carbon_tax}}
    \end{figure}
    
    
    
    \section{Summary of Key Recommendations}
    
    The primary recommendation is that a significant investment in wind capacity would be beneficial in all of the scenarios we have considered here.
    
    Even in the base model, extra wind capacity is likely to increase daily profits.
    
    In the case of fully reduced carbon dioxide limits, the only investment in current capacity that would result in a return to current profit levels is a 4 GW investment in wind.
    
    Under variable demand and wind scenarios, extra wind capacity would help reduce the reliance on other, more expensive sources, and hence increase profits.  Reducing variability of demand throughout the day (e.g. by offering incentives) would also be helpful.
    
    Expansion of hydro output capacity should be investigated for viability.
    
    
    \onecolumn
    
    \newpage
    
    \appendix
    
    % 
    %  APPENDIX A - PARAMS
    % 
    
    \section{Basic Model Setup and Parameters}\label{sec:params}

	Power output is measured in Megawatts (MW). Electricity (energy) generated is measured in Megawatt hours (MWh). The selling price of electricity is \textsterling 35 per MWh. Table \ref{table:powersources} in summarizes the output and costs of each of the six power sources. Also shown in Table \ref{table:emissionsconstraints} are the carbon dioxide and sulphur emissions, measured in standardized units per MWh of electricity generated from that power source.

	\begin{table}[H]
    \centering
    \begin{tabular}{lllllll}
    	\hline
    	~ & Gas & Coal & Nuclear & Wind & Hydro & Interconnect \\ \hline
        Max Output (MW) & 6000 & 6000 & 5000 & 5500 & 2000 & 20000 \\
        Run. Cost (\textsterling/MWh) & 40 & 30 & 50 & 2 & 5 & 100 \\
        Inc. Cost (\textsterling/MWh) & 80 & 60 & 1e+10 & 1 & 5 & 50 \\ \hline
        Carbon Em. & 0.0 & 0.4 & 0 & 0 & 0 & 0 \\
        Sulphur Em. & 0.8 & 1.2 & 0 & 0 & 0 & 0 \\ \hline
    \end{tabular}
    \caption{Properties of the various power sources\label{table:powersources}}
    \end{table}

	\begin{table}[H]
    \centering
    \begin{tabular}{lll}
    	\hline
    	~ & Carbon Dioxide & Sulphur \\ \hline
    	Daily Limit& 200000  & 30000  \\ \hline
    \end{tabular}
    \caption{Total permitted daily emissions, in standardized units\label{table:emissionsconstraints}}
    \end{table}


    
    % 
    %  APPENDIX B - FULL OUTPUT
    % 
    
    \section{Current Operations}\label{sec:app-currops}
    
    \begin{table}[H]
    	\centering
    	\begin{tabular}{ccccccc}
        	\hline
        	Period & Gas & Coal & Nuclear & Wind & Hydro & Interconnect \\ \hline
            1 & 5154 & 0 & 3846 & 5000 & 0 & 0 \\
            2 & 5921 & 3233 & 3846 & 5000 & 0 & 0 \\
            3 & 5921 & 4333 & 3846 & 5000 & 900 & 0 \\
            4 & 5921 & 5233 & 3846 & 5000 & 2000 & 0 \\
            5 & 5921 & 3233 & 3846 & 5000 & 0 & 0 \\
            6 & 5921 & 3233 & 3846 & 5000 & 0 & 0 \\ \hline
        \end{tabular}
    	\caption{Optimal generation schedule for the base scenario\label{table:basic-optgensched}}
    \end{table}
    
    
    
    \begin{table}[H]
    	\centering
    	\begin{tabular}{cccc}
        	\hline
        	Period & Discharge (MW) & Pump (MW) & Reserve (MWh) \\ \hline
            1 & 0 & 2000 & 12800 \\
            2 & 0 & 0 & 13600 \\
            3 & 900 & 0 & 9600 \\
            4 & 2000 & 0 & 3200 \\
            5 & 2000 & 0 & 0 \\
            6 & 0 & 0 & 800 \\ \hline
        \end{tabular}
    	\caption{Optimal hydro schedule for the base scenario\label{table:base-hydrosched}}
    \end{table}
    
    \begin{figure}[H]
    \centering
    \charthydro{output/base.csv}
    	\caption{Optimal hydro schedule for the base scenario\label{fig:base-hydrosched}}
    \end{figure}
    

    
    % 
    %  APPENDIX C
    % 
    
    \section{Variable Demand and Wind Availability}\label{sec:app-variable}
	
    \begin{table}[h!]
    	\centering
        \begin{tabular}{ccccccc}
        	\hline
        	Period & Gas & Coal & Nuclear & Wind & Hydro & Interconnect \\ \hline
        	1 & 5921 & 0 & 3846 & 6000 & 0 & 0 \\
            2 & 6000 & 4337 & 3846 & 6000 & 743 & 674 \\
            3 & 6000 & 4337 & 3846 & 6000 & 743 & 3074 \\
            4 & 6000 & 4337 & 3846 & 6000 & 2000 & 4217 \\
            5 & 6000 & 3754 & 3846 & 6000 & 183 & 4217 \\
            6 & 5921 & 1433 & 3846 & 6000 & 183 & 4217 \\ \hline
        \end{tabular}
        \caption{Optimal generation schedule for a 20\% increase in demand and wind\label{table:variable-optgensched}}
    \end{table}
    
    
    
    % 
    %  APPENDIX D
    % 
    
    \section{Reducing CO2}\label{sec:app-co2}
    
    
    \begin{table}[h]
    	\centering
    	\centering
        \begin{tabular}{cccccccc}
        	\hline
        	CO\textsubscript{2} Reduction (\%) & Gas & Coal & Nuclear & Wind & Hydro & Interconnect & Profit (\textsterling) \\ \hline
			   10 & 112500 & 75000 & 117300 & 120000 & 19200 & 0 & 1783667 \\
               15 & 125500 & 58000 & 120000 & 120000 & 14000 & 0 & 1553333 \\
               20 & 132000 & 45333 & 120000 & 120000 & 19200 & 7467 & 1000667 \\
               25 & 132000 & 37000 & 120000 & 120000 & 19200 & 15800 & 423286 \\
               30 & 132000 & 28667 & 120000 & 120000 & 19200 & 24133 & -155259 \\
               35 & 132000 & 20333 & 120000 & 120000 & 19200 & 32467 & -733963 \\
               40 & 132000 & 12000 & 120000 & 120000 & 19200 & 40800 & -1312667 \\
               45 & 132000 & 3667 & 120000 & 120000 & 19200 & 49133 & -1891370 \\
               50 & 125000 & 0 & 120000 & 120000 & 13600 & 58400 & -2491333 \\ \hline
        \end{tabular}
        \caption{Overall daily source utilisation composition and profit level by decreasing CO\textsubscript{2}\label{table:co2-rampdown}}
    \end{table}
    
    
    \begin{table}[H]
    	\centering
\begin{tabular}{cccccccc}
        	\hline
Add (MW) & Profit (\textsterling)       & Gas    & Coal & Nuclear & Wind   & Hydro & Interconnect \\ \hline
1000     & -2.49133e+06 & 125000 & 0    & 120000  & 120000 & 13600 & 58400        \\
1500     & -2.49133e+06 & 125000 & 0    & 120000  & 120000 & 13600 & 58400        \\
2000     & -2.49133e+06 & 125000 & 0    & 120000  & 120000 & 13600 & 58400        \\
2500     & -2.49133e+06 & 125000 & 0    & 120000  & 120000 & 13600 & 58400        \\
3000     & -2.49133e+06 & 125000 & 0    & 120000  & 120000 & 13600 & 58400        \\
3500     & -2.49133e+06 & 125000 & 0    & 120000  & 120000 & 13600 & 58400        \\
4000     & -2.49133e+06 & 125000 & 0    & 120000  & 120000 & 13600 & 58400        \\
4500     & -2.49133e+06 & 125000 & 0    & 120000  & 120000 & 13600 & 58400       \\ \hline 
\end{tabular}
        \caption{Source expansion by capacity for coal\label{table:co2-sexp-coal}}
    \end{table}
    
    \begin{table}[H]
    	\centering
\begin{tabular}{cccccccc}
        	\hline
Add (MW) & Profit (\textsterling)       & Gas    & Coal & Nuclear & Wind   & Hydro & Interconnect \\ \hline
1000     & -2.44633e+06 & 125000 & 0    & 120000  & 120000 & 9600  & 57400        \\
1500     & -2.44633e+06 & 125000 & 0    & 120000  & 120000 & 9600  & 57400        \\
2000     & -2.44633e+06 & 125000 & 0    & 120000  & 120000 & 9600  & 57400        \\
2500     & -2.44633e+06 & 125000 & 0    & 120000  & 120000 & 9600  & 57400        \\
3000     & -2.44633e+06 & 125000 & 0    & 120000  & 120000 & 9600  & 57400        \\
3500     & -2.44633e+06 & 125000 & 0    & 120000  & 120000 & 9600  & 57400        \\
4000     & -2.44633e+06 & 125000 & 0    & 120000  & 120000 & 9600  & 57400        \\
4500     & -2.44633e+06 & 125000 & 0    & 120000  & 120000 & 9600  & 57400        \\ \hline
\end{tabular}
        \caption{Source expansion by capacity for gas\label{table:co2-sexp-gas}}
    \end{table}
    
    \begin{table}[H]
    	\centering
\begin{tabular}{cccccccc}
        	\hline
Add (MW) & Profit (\textsterling)       & Gas    & Coal & Nuclear & Wind   & Hydro & Interconnect \\ \hline
1000     & -2.47333e+06 & 125000 & 0    & 120000  & 120000 & 13600 & 58400        \\
1500     & -2.47333e+06 & 125000 & 0    & 120000  & 120000 & 13600 & 58400        \\
2000     & -2.47333e+06 & 125000 & 0    & 120000  & 120000 & 13600 & 58400        \\
2500     & -2.47333e+06 & 125000 & 0    & 120000  & 120000 & 13600 & 58400        \\
3000     & -2.47333e+06 & 125000 & 0    & 120000  & 120000 & 13600 & 58400        \\
3500     & -2.47333e+06 & 125000 & 0    & 120000  & 120000 & 13600 & 58400        \\
4000     & -2.47333e+06 & 125000 & 0    & 120000  & 120000 & 13600 & 58400        \\
4500     & -2.47333e+06 & 125000 & 0    & 120000  & 120000 & 13600 & 58400        \\ \hline
\end{tabular}
        \caption{Source expansion by capacity for hydro\label{table:co2-sexp-hydro}}
    \end{table}
    
    \begin{table}[H]
    	\centering
\begin{tabular}{cccccccc}
        	\hline
Add (MW) & Profit (\textsterling)       & Gas    & Coal    & Nuclear & Wind   & Hydro & Interconnect \\ \hline
1000     & -1.38533e+06 & 125000 & 0       & 144000  & 120000 & 18400 & 35600        \\
1500     & -838407      & 123000 & 1333.33 & 156000  & 120000 & 19200 & 24466.7      \\
2000     & -295048      & 120000 & 3333.33 & 168000  & 120000 & 19200 & 13466.7      \\
2500     & 218500       & 115000 & 6666.67 & 180000  & 120000 & 19200 & 3133.33      \\
3000     & 487778       & 84800  & 26800   & 192000  & 120000 & 14400 & 0            \\
3500     & 627778       & 50600  & 49600   & 204000  & 120000 & 16800 & 0            \\
4000     & 767778       & 16400  & 72400   & 216000  & 120000 & 19200 & 0            \\
4500     & 768572       & 12500  & 75000   & 217733  & 119567 & 19200 & 0        \\ \hline
\end{tabular}
        \caption{Source expansion by capacity for nuclear\label{table:co2-sexp-nuclear}}
    \end{table}
    
    
    \begin{table}[H]
    	\centering
\begin{tabular}{cccccccc}
        	\hline
Add (MW) & Profit (\textsterling)       & Gas    & Coal    & Nuclear & Wind   & Hydro & Interconnect \\ \hline
1000     & -1.36233e+06 & 125000 & 0       & 120000  & 132000 & 16000 & 47000        \\
1500     & -797833      & 125000 & 0       & 120000  & 138000 & 17200 & 41300        \\
2000     & -233333      & 125000 & 0       & 120000  & 144000 & 18400 & 35600        \\
2500     & 329648       & 124500 & 333.333 & 120000  & 150000 & 19200 & 29966.7      \\
3000     & 889593       & 123000 & 1333.33 & 120000  & 156000 & 19200 & 24466.7      \\
3500     & 1.44954e+06  & 121500 & 2333.33 & 120000  & 162000 & 19200 & 18966.7      \\
4000     & 2.00895e+06  & 120000 & 3333.33 & 120000  & 168000 & 19200 & 13466.7      \\
4500     & 2.55493e+06  & 117500 & 5000    & 120000  & 174000 & 19200 & 8300        \\ \hline
\end{tabular}
        \caption{Source expansion by capacity for wind\label{table:co2-sexp-wind}}
    \end{table}
    
    
    \newgeometry{margin=1cm}
\begin{landscape}


    % 
    %  APPENDIX E
    % 
    
    \section{Analysis of Carbon Credits Market}\label{sec:app-carbon}
    

    \begin{table}[H]
    	\centering
\begin{tabular}{ccccccccccc}
\hline
Profit       & TotalRunningCost & TotalIncreaseCost & TotalEmissionCreditCost & Co2Price & Gas    & Coal  & Nuclear & Wind   & Hydro & Interconnect \\ 
\hline
2.05267e+06  & 1.2682e+07       & 385333            & 0                       & 0        & 139400 & 75000 & 90400   & 120000 & 19200 & 0            \\
1.64659e+06  & 1.2682e+07       & 385333            & 406080                  & 4        & 139400 & 75000 & 90400   & 120000 & 19200 & 0            \\
1.24051e+06  & 1.2682e+07       & 385333            & 812160                  & 8        & 139400 & 75000 & 90400   & 120000 & 19200 & 0            \\
834427       & 1.2682e+07       & 385333            & 1.21824e+06             & 12       & 139400 & 75000 & 90400   & 120000 & 19200 & 0            \\
515280       & 1.2834e+07       & 556000            & 1.21472e+06             & 16       & 107400 & 75000 & 120000  & 120000 & 9600  & 0            \\
211600       & 1.2834e+07       & 556000            & 1.5184e+06              & 20       & 107400 & 75000 & 120000  & 120000 & 9600  & 0            \\
-92080       & 1.2834e+07       & 556000            & 1.82208e+06             & 24       & 107400 & 75000 & 120000  & 120000 & 9600  & 0            \\
-394640      & 1.296e+07        & 570000            & 1.98464e+06             & 28       & 120000 & 62400 & 120000  & 120000 & 9600  & 0            \\
-678160      & 1.296e+07        & 570000            & 2.26816e+06             & 32       & 120000 & 62400 & 120000  & 120000 & 9600  & 0            \\
-961680      & 1.296e+07        & 570000            & 2.55168e+06             & 36       & 120000 & 62400 & 120000  & 120000 & 9600  & 0            \\
-1.2452e+06  & 1.296e+07        & 570000            & 2.8352e+06              & 40       & 120000 & 62400 & 120000  & 120000 & 9600  & 0            \\
-1.52424e+06 & 1.32e+07         & 410000            & 3.03424e+06             & 44       & 132000 & 52800 & 120000  & 120000 & 19200 & 0            \\
-1.80008e+06 & 1.32e+07         & 410000            & 3.31008e+06             & 48       & 132000 & 52800 & 120000  & 120000 & 19200 & 0            \\
-2.07592e+06 & 1.32e+07         & 410000            & 3.58592e+06             & 52       & 132000 & 52800 & 120000  & 120000 & 19200 & 0            \\
-2.35176e+06 & 1.32e+07         & 410000            & 3.86176e+06             & 56       & 132000 & 52800 & 120000  & 120000 & 19200 & 0            \\
-2.482e+06   & 1.6896e+07       & 370000            & 336000                  & 60       & 132000 & 0     & 120000  & 120000 & 19200 & 52800        \\
-2.482e+06   & 1.7328e+07       & 530000            & -256000                 & 64       & 120000 & 0     & 120000  & 120000 & 9600  & 62400        \\
-2.466e+06   & 1.7328e+07       & 530000            & -272000                 & 68       & 120000 & 0     & 120000  & 120000 & 9600  & 62400        \\
-2.45e+06    & 1.7328e+07       & 530000            & -288000                 & 72       & 120000 & 0     & 120000  & 120000 & 9600  & 62400        \\
-2.218e+06   & 2.4528e+07       & 410000            & -7.6e+06                & 76       & 0      & 0     & 120000  & 120000 & 9600  & 182400       \\
-1.818e+06   & 2.4528e+07       & 410000            & -8e+06                  & 80       & 0      & 0     & 120000  & 120000 & 9600  & 182400       \\
-1.418e+06   & 2.4528e+07       & 410000            & -8.4e+06                & 84       & 0      & 0     & 120000  & 120000 & 9600  & 182400       \\
-1.018e+06   & 2.4528e+07       & 410000            & -8.8e+06                & 88       & 0      & 0     & 120000  & 120000 & 9600  & 182400       \\
-618000      & 2.4528e+07       & 410000            & -9.2e+06                & 92       & 0      & 0     & 120000  & 120000 & 9600  & 182400       \\
-218000      & 2.4528e+07       & 410000            & -9.6e+06                & 96       & 0      & 0     & 120000  & 120000 & 9600  & 182400       \\
182000       & 2.4528e+07       & 410000            & -1e+07                  & 100      & 0      & 0     & 120000  & 120000 & 9600  & 182400      \\ \hline 
\end{tabular}
        \caption{Profit and optimal production schedule given CO\textsubscript{2} credit price.\label{table:carbon_credit}}
    \end{table}
    
    
\end{landscape}
\restoregeometry

      
    \newgeometry{margin=1cm}
\begin{landscape}

%
% Appendix
%    

\section{Carbon Tax} \label{sec:app-tax}
    \begin{table}[H]
    	\centering
\begin{tabular}{cccccccccccc}
\hline
Profit       & TotalRunningCost & TotalIncreaseCost & TotalEmissionCreditCost & Co2Price & Gas    & Coal  & Nuclear & Wind   & Hydro & Interconnect &  \\ 
\hline
2.05267e+06  & 1.2682e+07       & 385333            & 0                       & 0        & 139400 & 75000 & 90400   & 120000 & 19200 & 0            &  \\
1.24659e+06  & 1.2682e+07       & 385333            & 806080                  & 4        & 139400 & 75000 & 90400   & 120000 & 19200 & 0            &  \\
440507       & 1.2682e+07       & 385333            & 1.61216e+06             & 8        & 139400 & 75000 & 90400   & 120000 & 19200 & 0            &  \\
-365573      & 1.2682e+07       & 385333            & 2.41824e+06             & 12       & 139400 & 75000 & 90400   & 120000 & 19200 & 0            &  \\
-1.08472e+06 & 1.2834e+07       & 556000            & 2.81472e+06             & 16       & 107400 & 75000 & 120000  & 120000 & 9600  & 0            &  \\
-1.7884e+06  & 1.2834e+07       & 556000            & 3.5184e+06              & 20       & 107400 & 75000 & 120000  & 120000 & 9600  & 0            &  \\
-2.49208e+06 & 1.2834e+07       & 556000            & 4.22208e+06             & 24       & 107400 & 75000 & 120000  & 120000 & 9600  & 0            &  \\
-3.19464e+06 & 1.296e+07        & 570000            & 4.78464e+06             & 28       & 120000 & 62400 & 120000  & 120000 & 9600  & 0            &  \\
-3.87816e+06 & 1.296e+07        & 570000            & 5.46816e+06             & 32       & 120000 & 62400 & 120000  & 120000 & 9600  & 0            &  \\
-4.56168e+06 & 1.296e+07        & 570000            & 6.15168e+06             & 36       & 120000 & 62400 & 120000  & 120000 & 9600  & 0            &  \\
-5.2452e+06  & 1.296e+07        & 570000            & 6.8352e+06              & 40       & 120000 & 62400 & 120000  & 120000 & 9600  & 0            &  \\
-5.92424e+06 & 1.32e+07         & 410000            & 7.43424e+06             & 44       & 132000 & 52800 & 120000  & 120000 & 19200 & 0            &  \\
-6.60008e+06 & 1.32e+07         & 410000            & 8.11008e+06             & 48       & 132000 & 52800 & 120000  & 120000 & 19200 & 0            &  \\
-7.27592e+06 & 1.32e+07         & 410000            & 8.78592e+06             & 52       & 132000 & 52800 & 120000  & 120000 & 19200 & 0            &  \\
-7.95176e+06 & 1.32e+07         & 410000            & 9.46176e+06             & 56       & 132000 & 52800 & 120000  & 120000 & 19200 & 0            &  \\
-8.482e+06   & 1.6896e+07       & 370000            & 6.336e+06               & 60       & 132000 & 0     & 120000  & 120000 & 19200 & 52800        &  \\
-8.882e+06   & 1.7328e+07       & 530000            & 6.144e+06               & 64       & 120000 & 0     & 120000  & 120000 & 9600  & 62400        &  \\
-9.266e+06   & 1.7328e+07       & 530000            & 6.528e+06               & 68       & 120000 & 0     & 120000  & 120000 & 9600  & 62400        &  \\
-9.65e+06    & 1.7328e+07       & 530000            & 6.912e+06               & 72       & 120000 & 0     & 120000  & 120000 & 9600  & 62400        &  \\
-9.818e+06   & 2.4528e+07       & 410000            & 0                       & 76       & 0      & 0     & 120000  & 120000 & 9600  & 182400       &  \\
-9.818e+06   & 2.4528e+07       & 410000            & 0                       & 80       & 0      & 0     & 120000  & 120000 & 9600  & 182400       &  \\
-9.818e+06   & 2.4528e+07       & 410000            & 0                       & 84       & 0      & 0     & 120000  & 120000 & 9600  & 182400       &  \\
-9.818e+06   & 2.4528e+07       & 410000            & 0                       & 88       & 0      & 0     & 120000  & 120000 & 9600  & 182400       &  \\
-9.818e+06   & 2.4528e+07       & 410000            & 0                       & 92       & 0      & 0     & 120000  & 120000 & 9600  & 182400       &  \\
-9.818e+06   & 2.4528e+07       & 410000            & 0                       & 96       & 0      & 0     & 120000  & 120000 & 9600  & 182400       &  \\
-9.818e+06   & 2.4528e+07       & 410000            & 0                       & 100      & 0      & 0     & 120000  & 120000 & 9600  & 182400       &  \\ \hline
\end{tabular}
        \caption{Profit and optimal production schedule given CO\textsubscript{2} tax.\label{table:carbon_tax}}
    \end{table}
    
\end{landscape}
\restoregeometry
    
\end{document}
